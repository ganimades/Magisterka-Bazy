\section{Różnice implementacji baz danych}

 W~rozdziale zostały pokazane różnice i~podobieństwa w~implementacji poszczególnych rozwiązań bazodanowych. Przedstawiono fragmenty kodów wykonujące te same operacje przy użyciu różnych baz danych. Opisany też został stopień skomplikowania każdej z~operacji. 

\subsection{Modele danych}
Modele danych definiowane są różnie w~zależności od wykorzystanej bazy danych. W~przypadku Core Data używany jest edytor w~środowisku programistycznym XCode opisany w~drugim rozdziale pracy. Stworzony schemat bazy widoczny jest na rysunku \ref{fig: nosql_data_scheme}. Do poprawnego działania bazy Core Data wymagane jest wygenerowanie klas reprezentujących zaprojektowane tabele. Klasy te mogą zostać stworzone na kilka sposobów: ręcznie, za pomocą wbudowanego narzędzia w~XCode lub za pomocą dodatkowego programu typu ,,mogenerator". 

Bazy SQL przedstawione w~pracy używają standardowych poleceń SQL do stworzenia tabel i~relacji pomiędzy nimi. Przykład kodu źródłowego \ref{lis:create_SQL_table_code} prezentuje zapytania użyte w~stworzonym do testów  programie. 

\begin{code}[
		language=swift,
		caption={Polecenia tworzenia tabel w~SQLite i~FMDB},
		label={lis:create_SQL_table_code},
		escapechar=|
	]
    let createBookTableString = "CREATE TABLE IF NOT EXISTS Book (id INTEGER PRIMARY KEY AUTOINCREMENT, title TEXT, isbn TEXT, publishDate TEXT, publisher_id INTEGER, FOREIGN KEY(publisher_id) REFERENCES Publisher(id))"
    let createAuthorTableString = "CREATE TABLE IF NOT EXISTS Publisher (id INTEGER PRIMARY KEY AUTOINCREMENT, name TEXT)"
    let createPublisherTableString = "CREATE TABLE IF NOT EXISTS Publisher (id INTEGER PRIMARY KEY AUTOINCREMENT, name TEXT)"
\end{code}

\bigskip
W~przypadku dokumentowej bazy Realm kolekcje tworzone są na podstawie obiektów dziedziczących z~klasy \textit{Object} biblioteki Realm. Kod źródłowy \ref{lis:realm_object_code} pokazuje przykład klasy.

\begin{code}[
		language=swift,
		caption={Przykład obiektu bazy Realm},
		label={lis:realm_object_code},
	]
class RAuthor: Object {
    
    @objc dynamic var id: Int = 0
    @objc dynamic var firstName = ""
    @objc dynamic var lastName = ""    
    var books = List<RBook>()
    
    override static func primaryKey() -> String? {
        return "id"
    }
}
\end{code}
\bigskip

 W~liniach 3-5 przedstawione są zmienne przechowujące dane obiektu, linia 6 pokazuje listę obiektów \textit{RBook}. Lista ta jest odwzorowaniem relacji pomiędzy tablicami. Metoda \textit{primaryKey()} znajdująca się w~linii numer 10 oznacza pole odpowiadające za klucz główny danej klasy. 

Domyślna Baza Użytkownika, jak wspomniane zostało w~rozdziale 2.5, do konwersji obiektów wymaga rozszerzenia obiektów o~protokół \textit{NSCoding}. Przykład kodu klasy Domyślnej Bazy Użytkownika reprezentuje kod źródłowy \ref{lis:ud_object_code}. Tak jak w~przypadku Realm linie 2-4 reprezentują pola przechowujące dane za w~linii 5 przedstawiona jest lista identyfikatorów książek wydanych przez danego autora. Lista ta reprezentuje relacje pomiędzy obiektami. Metoda \textit{encode(with aCoder: NSCoder)} potrzebna jest do poprawnego zapisu obiektu w~pamięci zaś funkcja \textit{convenience init(coder aDecoder: NSCoder)} odpowiada za odczyt obiektu.

\begin{code}[
		language=swift,
		caption={Przykład obiektu Domyślnej Bazy Użytkownika},
		label={lis:ud_object_code},
	]
class UDAuthor: NSObject, NSCoding {
    var authorId: Int
    var firstName: String
    var lastName: String
    var books: [Int]
    
    init(authorId: Int, firstName: String, lastName: String, books: [Int]) {
        self.authorId = authorId
        self.firstName = firstName
        self.lastName = lastName
        self.books = books
    }
    
    func encode(with aCoder: NSCoder) {
        aCoder.encode(authorId, forKey: "authorId")
        aCoder.encode(firstName, forKey: "firstName")
        aCoder.encode(lastName, forKey: "lastName")
        aCoder.encode(books, forKey: "books")
    }
    
    required convenience init(coder aDecoder: NSCoder) {
        let authorId = aDecoder.decodeInteger(forKey: "authorId")
        let firstName = aDecoder.decodeObject(forKey: "firstName") as! String
        let lastName = aDecoder.decodeObject(forKey: "lastName") as! String
        let books = aDecoder.decodeObject(forKey: "books") as! [Int]
        
        self.init(authorId: authorId, firstName: firstName, lastName: lastName, books: books)
    }
}

\end{code}
\bigskip

Można zauważyć znaczące różnice w~tworzeniu struktury danych w~każdej z~przedstawionych baz. Core Data dzięki dobremu wsparciu ze strony XCode może wydawać się prosta w~zaimplementowaniu, lecz wymaga dobrej znajomosci całego framework-a. W~celu szybszej implementacji klas reprezentujących tabele, trzeba skorzystać z~oddzielnych narzędzi, co może przysporzyć dodatkowych problemów w~dalszych procesach utrzymywania oprogramowania. SQLite i~FMDB wymagają dobrej znajomości języka SQL, ale dobrze napisane polecenia są w~stanie posłużyć przez wiele lat, gdyż język SQL jest jednym z~najstabilniejszych i~najpopularniejszych rozwiązań bazodanowych. Realm wymaga stosunkowo małej ilości kodu i~umiejętności, aby stworzyć strukturę bazy. Klasy są bardzo przejrzyste i~zbliżone do zwykłych klas obiektów w~języku Swift. Domyślna Baza Użytkownika poza standardową deklaracją pól i~konstruktorów wymaga jeszcze zaimplementowania metod protokołu \textit{NSCoding}. Dodatkowo wymagane jest dodanie odpowiednich kluczy dla każdego z~pól obiektu. Ilość kodu i~kluczy uzależniona jest tutaj od ilości pól obiektu. Przy wielu tabelach może to stanowić znaczące problemy w~implementacji oraz utrzymaniu kodu. 

\subsection{Zapis danych}

Zapis danych w~przypadku Core Data przedstawiony jest w~przykładzie \ref{lis:core_data_save_code} znajdującym się poniżej.

\begin{code}[
		language=swift,
		caption={Przykład zapisu obiektu Core Data},
		label={lis:core_data_save_code},
	]
    private func insertPublishers(list: [Publisher]) {
        list.forEach { (publisher) in
            let publisherObject = CDWydawnictwo(context: stack.managedObjectContext)
            publisherObject.fill(publisher: publisher)
            guard let booksList = publisher.books else { return }
            booksList.forEach({ (bookId) in
                guard let book = featchBook(bookId: bookId) else { return }
                book.wydawnictwo = publisherObject
                publisherObject.ksiazki.adding(book)
            })
        }
        stack.saveContext()
    }
\end{code}
\bigskip

Dla każdego obiektu z~listy tworzony jest obiekt \textit{CDWydawnictwo} będący reprezentacją rekordu w~tabeli wydawnictwo. W~linii za pomocą funkcji \textit{fill} uzupełniane są pola obiektu. Następnie dla każdej książki wydanej przez wydawnictwo w~pętli pokazanej w~linii 8 odczytywany jest obiekt \textit{CDKsiazka} wcześniej już zapisany w~bazie. Po odczytaniu obiektu następuje ustawienie relacji książka - wydawnictwo, co reprezentują linie 10 i~11 w~przedstawionym przykładzie. Po wykonaniu opisanych operacji w~linii numer 15 następuje zapis stanu bazy Core Data. 

Podczas użycia baz SQLIte i~FMDB, aby zapisać obiekt do bazy, należy w~pierwszej kolejności zdefiniować odpowiednie zapytania. Aby poprawnie wprowadzać dane do przedstawionej w~pracy bazie testowej, zostały użyte zapytania SQL widoczne w~przykładzie \ref{lis:insert_sql_query}. 

\begin{code}[
		language=swift,
		caption={Zapytania SQL do wprowadzania danych},
		label={lis:insert_sql_query},
	]
    private let insertBookDataQueryString = "INSERT INTO Book (title, isbn, publishDate) VALUES (?, ?, ?)"
    private let insertAuthorDataQueryString = "INSERT INTO Author (firstName, lastName) VALUES (?, ?)"
    private let insertPublisherDataQueryString = "INSERT INTO Publisher (name) VALUES (?)"
    private let insertAuthorsBooksDataQueryString = "INSERT INTO authors_books (book_id, author_id) VALUES (?, ?)"
\end{code}
\bigskip

Kolejnym krokiem jest odpowiednie użycie przedstawionych powyżej zapytań oraz przypisanie odpowiednich pól obiektu do zapytania. Czynności te dla bazy SQLite zostały pokazane we fragmencie kodu \ref{lis:sqlite_save_code}. Przykład pokazuję zapis obiektu \textit{Książka}. Aby zapisać obiekt, niezbędna jest zmienna typu \textit{OpaquePointer} reprezentująca obiekt zapytania do bazy danych. Za pomocą funkcji widocznej w~4 linii \textit{sqlite3\_prepare\_v2} zapytanie z~formy ciągu znaków konwertowane jest na obiekt. Następnie w~liniach 8-11 za pomocą funkcji \textit{sqlite3\_bind\_text} pola obiektu \textit{Książka} przypisywane są do zapytania. W~linii 13 przy użyciu \textit{sqlite3\_step} następuje wykonanie zapytania i~zapis obiektu do tablicy. Funkcja \textit{sqlite3\_reset} czyści przypisane w~zapytaniu zmienne i~umożliwia ponowne załadowanie nowych danych do zapytania. W~linii 20 przykładu następuje zapis stanu bazy danych przy użyciu \textit{sqlite3\_finalize}.

\begin{code}[
		language=swift,
		caption={Przykład zapisu obiektu SQLIte},
		label={lis:sqlite_save_code},
	]
    private func insertBooks(list: [Book]) {
        var insertDataStatement: OpaquePointer?
        
        if sqlite3_prepare_v2(db, insertBookDataQueryString, -1, &insertDataStatement, nil) != SQLITE_OK {
            print("error create database statement")
        }
        
        list.forEach { (book) in
            sqlite3_bind_text(insertDataStatement!, 1, (book.title as! NSString).utf8String, -1, nil)
            sqlite3_bind_text(insertDataStatement!, 2, (book.isbn as! NSString).utf8String, -1, nil)
            sqlite3_bind_text(insertDataStatement!, 3, (book.stringDate as NSString).utf8String, -1, nil)
            
            if sqlite3_step(insertDataStatement!) != SQLITE_DONE {
                print("Could not insert row into Book table")
            }
            
            sqlite3_reset(insertDataStatement!)
        }
        
        sqlite3_finalize(insertDataStatement)
    }
\end{code}
\bigskip

Przy użyciu biblioteki FMDB użycie SQLite jest znacznie łatwiejsze. Bliźniacza metoda do zapisu obiektów \textit{Książka} została przedstawiona w~przykładzie \ref{lis:fmdb_save_code}. Proces zapisu obiektu jest znacząco ułatwiony. W~linii 2 poprzez funkcje \textit{open()} otwierane jest połączenie z~bazą danych. Następnie w~pętli wykonywana jest operacja \textit{executeUpdate}, która przyjmuje jako parametr zapytanie do bazy i~listę pól obiektu. Po zapisaniu wszystkich obiektów do bazy w~linii numer 15 wywoływana jest funkcja \textit{close} służąca do zamknięcia połączenia z~bazą danych. 

\begin{code}[
		language=swift,
		caption={Przykład zapisu obiektu FMDB},
		label={lis:fmdb_save_code},
	]
    private func insertBooks(list: [UDBook]) {
        guard database.open() else {
            print("Unable to open database")
            return
        }
        
        list.forEach({ (book) in
            do {
                try! database.executeUpdate(insertBookDataQueryString, values: [book.title, book.isbn, book.stringDate])
            } catch {
                print("failed: error.localizedDescription")
            }
        })
        
        database.close()
    }
\end{code}
\bigskip

Jedną z~najprostszych implementacji zapisu danych posiada Domyślna Baza Użytkownika. Kod wykonujący to samo zadanie co przytoczone wcześniej przykłady reprezentuje fragment kodu \ref{lis:ud_save_code}. Dzięki implementacji protokołu \textit{NSCoding} możliwe jest szybkie kodowanie i~dekodowanie obiektów. W~celu uzyskania formatu danych pozwalającego na zapis w~Domyślnej bazie użytkownika użyta została metoda \textit{NSKeyedArchiver.archivedData} przekształcająca listę obiektów na typ \textit{Data}, a~następnie dane zostały zapisane za pomocą funkcji \textit{userDefaults.set}, a~w ostatniej linii za pomocą metody \textit{synchronize} zapisany został stan bazy. 

\begin{code}[
		language=swift,
		caption={Przykład zapisu obiektu User Defaults},
		label={lis:ud_save_code},
	]
    private func insertBooks(list: [UDBook]) {
        let encodedData = NSKeyedArchiver.archivedData(withRootObject: list)
        userDefaults.set(encodedData, forKey: booksKey)
        userDefaults.synchronize()
    }
\end{code}
\bigskip

Biblioteka Realm także posiada prosty interfejs zapisu danych. Metoda wykonująca to zadanie przedstawiona została poniżej w~przykładzie \ref{lis:realm_save_code}. Cała operacja sprowadza się do stworzenia bloku zapisu \textit{write} i~wywołaniu metody \textit{add} z~listą obiektów przeznaczoną do zapisu. 

\begin{code}[
		language=swift,
		caption={Przykład zapisu obiektu Realm},
		label={lis:realm_save_code},
	]
    private func insertBooks(list: [RBook]) {
        try! realm.write {
            realm.add(list)
        }
    }
\end{code}
\bigskip

Można zauważyć, że najmniej wymagający interfejs do zapisu danych posiada biblioteka Realm. Składa się on praktycznie z~dwóch metod. Bardzo prosty zapis danych posiada także Domyślna Baza Użytkownika lecz w~jej przypadku należy dodatkowo w~każdym obiekcie implementować protokół \textit{NSCoding}. Jedną z~bardziej wymagających bibliotek jest Core Data. Zapis danych wymaga tutaj większej ilości kodu, a~także konwersji zapisywanego obiektu do obiektu reprezentującego tablice bazy. Należy też zadbać o~zapisywanie kontekstu bazy w~odpowiednich momentach. Najbardziej wymagające są bazy SQL. SQLite i~FMDB wymagają sformułowania odpowiednich zapytań SQL, dodatkowo w~przypadku SQLite implementacja wymaga pojedynczego przypisywania pól obiektu do stworzonego zapytania. Problem ten znika przy użyciu FMDB, lecz w~większości przypadków to rozwiązanie pomimo ułatwień w~implementacji okazuję się działać znacznie wolniej od SQLite. 

\subsection{Odczyt danych}
Odczyt danych zostanie porównany na podstawie funkcji użytej podczas testu przedstawionego w~rozdziale 6.2 polegającego na wyszukaniu wszystkich autorów o~imieniu ,,Diena". 

Wybrany przykład funkcja dla Core Data został przedstawiony poniżej w~kodzie źródłowym \ref{lis:core_data_read_code}. Do odczytu danych należy zainicjalizować obiekt \textit{NSFetchRequest} z~odpowiednią nazwą tablicy, na której przeprowadzona ma zostać operacja. Następnie tworzony jest \textit{NSPredicate}, który odpowiada za odpowiednie wyszukanie rezultatu zapytania, w~tym przypadku wybraniu autorów o~konkretnym imieniu. Po prawidłowej konfiguracji operacja jest wykonywana za pomocą funkcji \textit{executeFetchRequest}, która jako parametr przyjmuje \textit{NSFetchRequest} i~kontekst bazy danych.

\begin{code}[
		language=swift,
		caption={Przykład odczytu danych Core Data},
		label={lis:core_data_read_code},
	]
    func getAuthorsByName() {
        let name = "Diena"
        
        let request = NSFetchRequest<NSFetchRequestResult>(entityName: CDAuthor.entityName())
        let predicate = NSPredicate(format: "name == %@", name)
        
        request.predicate = predicate
        
        guard let result = stack.executeFetchRequest(request: request, inContext: stack.managedObjectContext) else { return }
    }
\end{code}
\bigskip

Inny sposób przeszukiwania danych oferuje Domyślna Baza Użytkownika. Funkcja wykonująca te samo zadanie została przedstawiona w~przykładzie \ref{lis:ud_read_code}.

\begin{code}[
		language=swift,
		caption={Przykład odczytu danych User Defaults},
		label={lis:ud_read_code},
	]
    func getAuthorsByName() {
        let name = "Diena"
        var result = [UDAuthor]()
        
        if  let authorsData = userDefaults.data(forKey: authorsKey) {
            let authorsList = NSKeyedUnarchiver.unarchiveObject(with: authorsData) as! [UDAuthor]
            result = authorsList.filter { firstName == name }
        }
    }
\end{code}
\bigskip

 W~5 linii za pomocą zdefiniowanego klucza dla tabeli odczytywane są wszystkie dane dla tablicy Autor. Następnie w~linii numer 6 następuje konwersja surowego typu danych na obiekty. Kolejnym krokiem jest przeszukanie listy i~wybranie autorów o~podanym imieniu. 

Bazy SQL do odczytania danych wymagały zdefiniowana zapytania widocznego poniżej. Zapytanie te zostało użyte dla bazy SQLite i~FMDB.

\begin{code}[
		language=swift,
		caption={Zapytanie SQL do odczytu danych},
		label={lis:sql_read_query_code},
	]
let selectAuthorByNameQueryString = "SELECT id, firstName, lastName FROM Author a~WHERE a.firstName = ?"
\end{code}
\bigskip

 W~przypadku użycia SQLite odczyt danych przeprowadzony był poprzez poprzez funkcje widoczna w~fragmencie kodu \ref{lis:sqlite_read_code}

\begin{code}[
		language=swift,
		caption={Przykład odczytu danych SQLite},
		label={lis:sqlite_read_code},
	]
    func getAuthorsByName() {
        let name = "Diena"
        var result = [Author]()
        var stmtAuthors: OpaquePointer?
        
        if sqlite3_prepare(db, selectAuthorByNameQueryString, -1, &stmtAuthors, nil) != SQLITE_OK {
            let errmsg = String(cString: sqlite3_errmsg(db)!)
            print("error preparing insert: errmsg")
            return
        }
        
        sqlite3_bind_text(stmtAuthors!, 1, (name as! NSString).utf8String, -1, nil)
        
        while(sqlite3_step(stmtAuthors) == SQLITE_ROW) {
            let id = Int(sqlite3_column_int(stmtAuthors, 0))
            let firstName = String(cString: sqlite3_column_text(stmtAuthors, 1))
            let lastName = String(cString: sqlite3_column_text(stmtAuthors, 2))
            
            result.append(Author(authorId: id, firstName: firstName, lastName: lastName))
        }
    }
\end{code}
\bigskip

Ponownie w~przypadku SQLite wcześniej zdefiniowane zapytanie zostaje przypisane do obiektu typu \textit{OpaquePointer} za pomocą funkcji \textit{sqlite3\_prepare}. Następnie w~linii 12 do zapytania zostaje przypisany parametr oznaczony za pomocą "?" w~zapytaniu. W~linii numer 14 za pomocą funkcji \textit{sqlite3\_step} w~pętli odczytywane są kolejne rekordy będące wynikiem zapytania. Każde z~pól obiektu musi zostać odczytane i~sformatowane na odpowiedni typ. 

Operacja odczytu została ułatwia FMDB w~stosunku do SQLite. Funkcja wykonująca tą samą operację przy użyciu bazy FMDB została przedstawiona poniżej. 

\begin{code}[
		language=swift,
		caption={Przykład odczytu danych FMDB},
		label={lis:fmdb_read_code},
	]
func getAuthorsByName() {
        let name = "Diena"
        var authorsList = [Author]()
        
        guard database.open() else {
            print("Unable to open database")
            return
        }
        
        do {
            let result = try database.executeQuery(selectAuthorByNameQueryString, values: [name])
            
            while(result.next()) {
                let id = Int(result.int(forColumnIndex: 0))
                let firstName = result.string(forColumnIndex: 1)
                let lastName = result.string(forColumnIndex: 2)
                
                authorsList.append(Author(authorId: id, firstName: firstName, lastName: lastName))
            }
        } catch {
            print("failed: (error.localizedDescription)")
        }
        
        database.close()
    }
\end{code}
\bigskip

FMDB po otworzeniu połączenia z~bazą umożliwia wykonanie zapytania za pomocą funkcji \textit{executeQuery} i~podaniu argumentów zapytania jako parametry funkcji. Następnie można iterować po kolejnych rekordach będących wynikiem zapytania. Rezultaty nie wymagają ponownego parsowania na odpowiednie typy. Po zakończeniu operacji należy pamiętać o~zamknięciu połączenia z~bazą danych za pomocą funkcji \textit{close}. 

Jedną z~najprostszych implementacji posiada biblioteka Realm. Przykład reprezentuje fragment kodu \ref{lis:realm_read_code}.

\begin{code}[
		language=swift,
		caption={Przykład odczytu danych Realm},
		label={lis:realm_read_code},
	]
    func getAuthorsByName() {
        let name = "Diena"
        let predicate = NSPredicate(format: "firstName = %@", name)
        let result = realm.objects(RAuthor.self).filter(predicate)
    }
\end{code}
\bigskip

Realm tak samo jak Core Data używa \textit{NSPredicate} do sformułowania zapytania odczytu danych. W~linii 4 za pomocą funkcji \textit{object()}, podając odniesienie do klasy, której rezultat chcemy otrzymać, istnieje możliwość odczytania wszystkich rekordów. Następnie za pomocą funkcji \textit{filter}, której argumentem jest wcześniej zdefiniowany predykant, następuje filtrowanie wyniku. 

Odczyt danych jest inny dla każdej z~bibliotek. Core Data pomimo dość prostej składni tworzenia zapytań posiada często problematyczne w~użyciu sformułowania predykantów. Domyślna Baza Użytkownika operuje jedynie na funkcjach filtrowania oferowanych przez środowisko, przez co często operacje są znacznie wolniejsze niż w~przypadku innych rozwiązań bazodanowych. Dużym problemem jest też sytuacja kiedy wyszukanie rezultatu wymaga przeszukania relacji lub filtrowania wielu parametrów. W~takich przypadkach wymagane jest stworzenie wielu funkcji, co niesie za sobą duży nakład kodu. SQLite posiada problematyczne przypisywanie argumentów do zapytania a~także wymaga parsowania odczytanych pól na odpowiednie typy. FMDB rozwiązuje te problemy, lecz ilość kodu potrzebnego na zaimplementowanie jest zbliżona do implementacji SQLite. Jeden z~najprostszych sposobów implementacji posiada Realm. Odczytanie danych sprowadza się jedynie do kilku linii kodu. Oferuje on także możliwość używania predykatów tak samo jak w~przypadku Core Data, a~także zwykłego filtrowania wyniku jak Domyślna Baza Użytkownika.


\subsection{Usuwanie danych}

Usuwanie danych zostanie porównane na podstawie funkcji użytej podczas testu przedstawionego w~rozdziale 6.3 polegającego na usunięciu wszystkich autorów, którzy wydali 3 książki.

 Kod źródłowy \ref{lis:core_data_delete_code} przedstawia przykład usuwania danych przy użyciu biblioteki Core Data. W~usuwaniu danych tak samo, jak podczas odczytu tworzony jest \textit{NSFetchRequestResult}, za pośrednictwem którego odczytywane są dane z~określonej tabeli. Następnie otrzymany rezultat jest filtrowany. Po przefiltrowaniu w~linii 9 wykonywana jest operacja usuwania wybranych rekordów. Standardowo po wszystkich operacjach następuje zapis stanu bazy danych.
 
\begin{code}[
		language=swift,
		caption={Przykład usuwania danych Core Data},
		label={lis:core_data_delete_code},
	]
    func removeAuthorsByBooks() {
        let requestAuthors = NSFetchRequest<NSFetchRequestResult>(entityName: CDAutor.entityName())
        
        guard var result = stack.executeFetchRequest(request: requestAuthors, inContext: stack.managedObjectContext) as? [CDAutor] else { return }
        result = result.filter { 0.ksiazki.count == 3 }
        
        result.forEach { (author) in
            stack.managedObjectContext.delete(author)
        }
        
        stack.saveContext()
    }
 \end{code}
    \bigskip
    
Kod źródłowy \ref{lis:ud_delete_code} reprezentuje usuwanie danych w~Domyślnej Bazie Użytkownika. W~przykładzie widać słabe strony tego rozwiązania bazodanowego. W~linii 4 następuje filtrowanie listy zapisanych rekordów w~taki sposób, aby wyszukać identyfikatory autorów, których należy usunąć. W~linii 5 wykonywane jest ponowne przeszukanie listy rekordów i~usunięcie wybranych danych. W~przypadku złożonych operacji podczas używania Domyślnej Bazie Użytkownika często istnieje potrzeba przeszukiwania kilkukrotnie więcej niż jednej listy. Często jest to problematyczne w~zaimplementowaniu oraz znacząco spowalnia aplikacje.
    
\begin{code}[
		language=swift,
		caption={Przykład usuwania danych User Defaults},
		label={lis:ud_delete_code},
	]
        func removeAuthorsByBooks() {
        if  let authorsData = userDefaults.data(forKey: authorsKey) {
            let authorsList = NSKeyedUnarchiver.unarchiveObject(with: authorsData) as! [UDAuthor]
            let resultIdsToRemove = authorsList.filter { 0.books.count == 3 }.map { 0.authorId }
            let result = authorsList.filter { !resultIdsToRemove.contains(0.authorId) }
            let encodedData = NSKeyedArchiver.archivedData(withRootObject: result)
            userDefaults.set(encodedData, forKey: authorsKey)
            userDefaults.synchronize()
        }
    }
\end{code}
\bigskip

Bazy SQL do usuwania danych wymagały zdefiniowana zapytania widocznego poniżej. Zapytanie te zostało użyte dla bazy SQLite i~FMDB.

\begin{code}[
		language=swift,
		caption={Zapytanie SQL do usuwania danych},
		label={lis:sql_delete_query_code},
	]
let deleteAuthorsByBooks = "DELETE FROM Author WHERE EXISTS (SELECT authorId FROM ( SELECT Author.id as authorId, COUNT(authors_books.book_id) as booksCount FROM Author, authors_books WHERE Author.id = authors_books.author_id GROUP BY Author.id) WHERE booksCount = 3)"
\end{code}
\bigskip

Przykłady kodów źródłowych \ref{lis:sqlite_delete_code} (SQLite) i~\ref{lis:fmdb_delete_code} (FMDB) pokazują, że implementacja tego samego zadania nie różni się znacząco. Największą różnicę pomiędzy nimi stanowi składnia. SQLite posiada składnię pokrewną dla języka C zaś biblioteka FMDB pomimo wykorzystywania SQLite udostępnia wszystkie funkcja charakterystyczne dla języka Swift. 

\begin{code}[
		language=swift,
		caption={Przykład usuwania danych SQLite},
		label={lis:sqlite_delete_code},
	]
    func removeAuthorsByBooks() {
        var stmtAuthors: OpaquePointer?
        
        if sqlite3_prepare(db, deleteAuthorsByBooks, -1, &stmtAuthors, nil) != SQLITE_OK {
            let errmsg = String(cString: sqlite3_errmsg(db)!)
            print("error preparing insert: (errmsg)")
            return
        }
        
        sqlite3_step(stmtAuthors!)
        sqlite3_finalize(stmtAuthors)
    }
\end{code}
\bigskip

\begin{code}[
		language=swift,
		caption={Przykład usuwania danych FMDB},
		label={lis:fmdb_delete_code},
	]
    func removeAuthorsByBooks() {
        guard database.open() else {
            print("Unable to open database")
            return
        }

        do {
            try database.executeStatements(deleteAuthorsByBooks)
        } catch {
            print("failed: (error.localizedDescription)")
        }
        
        database.close()
    }
\end{code}
\bigskip

Najmniej wymagający interfejs posługiwania się bazą danych zaprezentowany został w~przykładzie \ref{lis:realm_delete_code}. Biblioteka Realm wymaga jedynie przefiltrowania listy rekordów, a~następnie przekazanie rezultatu jako argument funkcji \textit{delete}. 

\begin{code}[
		language=swift,
		caption={Przykład usuwania danych Realm},
		label={lis:realm_delete_code},
	]
    func removeAuthorsByBooks() {
        try! realm.write {
            let authors = realm.objects(RAuthor.self).filter { 0.books.count == 3 }
            realm.delete(authors)
        }
    }
\end{code}
\bigskip

Przykłady usuwania danych bardzo dobrze pokazują, w~jakim stopniu różni się sposób implementacji operacji na danych w~przypadku różnych bibliotek. Core Data posiada swój charakterystyczny sposób przeprowadzania operacji za pomocą \textit{NSFetchRequest}. Implementacja SQLite i~FMDB są do siebie zbliżone. W~przypadku SQLite implementacja może wydawać się trudniejsza ze względu na interfejs C biblioteki. Domyślna Baza Użytkownika jest doskonałym rozwiązaniem do prostych operacji. Podczas skomplikowanych operacji na danych ilość kodu i~czas poświęcony na implementacje wzrasta. Najmniej wymagającą biblioteką jest dokumentowa baza Realm. Nie wymaga ona znaczących umiejętności od programisty, a~także przeprowadzenie operacji na danych sprowadza się najczęściej do wywołania paru metod udostępnionych przez bibliotekę.