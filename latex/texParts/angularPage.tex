\section{Angular}
	15 września 2016 roku po blisko półtora roku prac Google wydało wzorowaną na poprzedniku, przepisaną od nowa nową wersję - Angular 2. Określana słowami \textit{One framework. Mobile \& desktop.} platforma programistyczna do tworzenia aplikacji internetowych na urządzenia mobilne i desktopowe z wykorzystaniem języka TypeScript. Implementuje podstawowe i opcjonalne funkcjonalności jako zbiór bibliotek gotowych do zaimportowania. W kolejnych podrozdziałach omówione zostaną podstawowe struktury i sposoby tworzenia aplikacji.
	
	\subsection{Nazewnictwo i wersjonowanie}  
	Początkowo Angular miał pozostać w wersji 2, tak samo jak AngularJS pozostał w wersji 1, dlatego właśnie początkowa nazwa - Angular 2 \cite{angular-versioning}. Jednak miesiąc po oficjalnej publikacji twórcy ogłosili zmianę sposobu wersjonowania na SEMVER (ang. \textit{Semantic Versioning}). Głównym celem SEMVER jest nadanie znaczenia numerom wersji. Pozwala to na zobrazowanie jak duże zmiany wprowadzono, a nawet umożliwienie narzędziom takim jak NPM (ang. \textit{Node Package Manager}) na aktualizowanie \blockquote{paczek} w automatyczny i bezpieczny sposób.\par
	
	Każda wersja składa się z 3 liczb. Przykładem jest \textit{2.5.3}. Kiedy naprawiony zostaje jakiś \textit{bug} zwiększana jest ostatnia liczba. Kiedy dodana jest nowa funkcjonalność zwiększana jest środkowa liczba. Kiedy wydanie posiada \textit{breaking change}, to znaczy zmianę taką, która zmusza użytkownika do dostosowania wcześniej działającego programu do nowej wersji, zwiększana jest pierwsza liczba.\par
	
	Dlatego właśnie początkowa nazwa Angular 2 od tamtej pory przekształciła się w Angular. Na czas pisania tej pracy najnowszą wersją jest 6.0.0.
	
	\subsection{RxJS}
	Jednym z założeń przepisania frameworku była zmiana sposobu detekcji zmian. Zdecydowano się na programowanie reaktywne. Jest to asynchroniczny paradygmat programowania dotyczący strumieni danych i propagacji zmian. Do tego celu użyto biblioteki RxJS (ang. \textit{Reactive Extensions for JavaScript}), która zapewnia implementację typu \textit{Observable} oraz dużą ilość operatorów (ponad 150), które umożliwiają dowolną manipulację strumieniami danych, ich łączenie, czy odwoływanie w czasie. W kodzie źródłowym \ref{lis:rxjs-example} przedstawiony został przykład wypisania na konsolę potęg liczb, które są mniejsze od 5.
	
	\begin{code}[
		language=javascript,
		caption={Przykład użycia RxJS (źródło: opracowanie własne)},
		label={lis:rxjs-example},
	]
Observable.of(1, 2, 3)
    .map(x => x * x)
    .filter(x => x < 5)
    .subscribe(x => console.log(x))
// Wynik:
// 1
// 4
	\end{code}	
	
	\subsection{TypeScript}
	Innym z założeń nowego frameworku, było to, aby był skalowalny. Bez względu na rozmiar aplikacji czytelność kodu nie powinna być skomplikowana. Jednym z takich problemów była trudność określenia typów zmiennych. W dużych programach bardzo łatwo popełnić błąd w nazwie pola obiektu lub pomylić jego typ, a ów problem uwidaczniał się dopiero w trakcie wykonania kodu. Z pomocą przyszedł, stworzony przez Microsoft, TypeScript \cite{typescript-doc}.\par
	
	TypeScript jest nadzbiorem JavaScript, to znaczy, że każdy poprawny program w JavaScript jest poprawnym programem w TypeScript, który pozwala na dodawanie statycznego typowania. Jest transpilowany do dowolnych wersji JavaScript, dzięki czemu funkcjonalności z najnowszych wersji ECMAScript nie są już przeszkodą w używaniu na starszych przeglądarkach.\par
	
	\begin{code}[
		language=javascript,
		caption={Przykład wykrycia błędu w TypeScript (źródło: opracowanie własne)},
		label={lis:typescript-example},
	]
function sayHello(name: string) {
  console.log("Hello " + name)
}

sayHello({ name: "John" })
	\end{code}		
	
	Przykład wykrycia błędu pokazano w kodzie źródłowym \ref{lis:typescript-example}. Funkcja \textit{sayHello} przyjmuje imię jako parametr i wita użytkownika komunikatem wyświetlonym w konsoli. Początkowo program wydaje się być poprawny, ale tak nie jest. Należy zwrócić uwagę na to, że funkcja przyjmuje nazwę jako \textit{string}, natomiast wywołana została z obiektem. W JavaScript problem zauważony zostałby dopiero po uruchomieniu programu. W przypadku TypeScript już sam kompilator zwraca błąd: \textit{error TS2345: Argument of type '{ name: string; }' is not assignable to parameter of type 'string'.} 
	
	\subsection{Kompilacja AOT}	
	Aplikacja Angularowa w większości składa się z komponentów i ich szablonów HTML. Przed wyrenderowaniem aplikacji przez przeglądarkę komponenty i szablony są tramspilowane przez kompilator Angulara do wykonywalnego kodu JavaScript. Angular oferuje 2 metody kompilacji: JIT (ang. \textit{Just-in-Time}) oraz AOT (ang. \textit{Ahead-of-Time}) \cite{angular-aot}.\par
	JIT, jak sama nazwa wskazuje, kompiluje aplikację dopiero w momencie uruchomienia w przeglądarce. Konsekwencją tego jest to, że oprócz samej aplikacji użytkownik musi pobrać również kompilator, który waży około 300kB. Może się to nie wydawać dużo, ale w erze urządzeń mobilnych każdy kilobajt się liczy. Oczywiście kompilowanie w trakcie wykonania posiada kolejny minus w postaci czasu. Im większy czas oczekiwania na interakcję z aplikacją, tym mniejsza ilość odwiedzających użytkowników, dlatego tak ważna jest jak najszybsza gotowość do użycia.\par
	AOT kompiluje aplikację już w trakcie budowania, a co za tym idzie użytkownik nie potrzebuje już pobierać kompilatora, przez co zaoszczędzi transfer danych i czas. Kolejnym plusem jest dołączenie szablonów HTML i arkuszów stylów CSS bezpośrednio do kodu JavaScript, dzięki czemu eliminuje się wysyłanie zapytań po te pliki. Kompilator AOT również wykrywa i zgłasza błędy wiązania szablonu HTML podczas kompilacji, zanim użytkownicy będą mogli je zobaczyć.
	
	\subsection{Podział na warstwy - NativeScript} \label{sec:angular-layers}
	AngularJS mógł być uruchomiony tylko w środowisku przeglądarki przez to, że był zależny od jQuery. Jednym z celów nowej wersji była możliwość uruchomienia aplikacji w różnych środowiskach. Dlatego zdecydowano się na podział na 2 warstwy \cite{angular-layers}: warstwę aplikacji oraz warstwę renderującą. Warstwa aplikacji zawiera API (ang. \textit{Application Programmin Interface}) oraz środowisko uruchomienia, z którym kod aplikacji komunikuje się bezpośrednio, a warstwa renderująca określa sposób, jak aktualizować UI (ang. \textit{User Interface}). Taka architektura warstwowa umożliwia działanie aplikacji na różnych platformach, tym samym nie zmieniając sposobu jej tworzenia. Aby stworzyć własną warstwę renderującą należy zaimplementować szereg interfejsów, w których struktury Angularowe tłumaczone są na język zrozumiały dla platformy, na której aplikacja miałaby zostać uruchomiona.\par
	NativeScript jest frameworkiem, który pozwala tworzyć natywne aplikacje mobilne używając Angulara, TypeScriptu lub JavaScriptu. Dzięki posiadaniu własnej warstwy renderującej, na podstawie kodu XML i komponentów renderuje elementy tak, jak miałoby to miejsce w przypadku kodu w Javie na Androida, lub w Swift na iOS.
	
	\subsection{Modularność}
	JavaScript i Angular używają modułów w celu organizacji struktury kodu i chociaż robią to w różny sposób, Angular używa obu tych technik.
	
	\subsubsection*{Moduły JavaScript}
	W JavaScript moduły to pojedyncze pliki z poprawnym kodem JavaScript. Aby sprawić, by pozostałe pliki mogły używać tego kodu wystarczy dopisać słowo kluczowe \textit{export} przed takimi strukturami, jak: funkcje, klasy, stałe, itp. Sposób ten przedstawiono w kodzie źródłowym \ref{lis:javascript-export-import} w linijce \ref{line:js-module-export}. Aby móc korzystać z kodu innych plików należy użyć schematu zaprezentowanego w linijce \ref{line:js-module-import}. Takie rozwiązanie zapobiega między innymi kolizji nazw oraz przypadkowym zmiennym globalnym.
	
	\begin{code}[
		language=javascript,
		caption={Stosowanie modułów JavaScript (źródło: \cite{javascript-module})},
		label={lis:javascript-export-import},
		escapechar=|
	]
// app.component.js
export class AppComponent { ... } |\label{line:js-module-export}|

// main.js
import { AppComponent } from './app.component'; |\label{line:js-module-import}|
	\end{code}	
	
	\subsubsection*{Moduły Angular}
	W Angular moduły są to klasy opisane dekoratorem \textit{@NgModule}. Zawierają on metadane na temat tego co dany moduł eksportuje, co importuje, jakie nowe struktury tworzy, od czego jest zależny, itp. W kodzie źródłowym \ref{lis:angular-module} przedstawiono deklarację modułu \textit{AppModule}, który deklaruje nowy komponent \textit{AppComponent} oraz importuje angularowy, zewnętrzny moduł przeglądarki \textit{BrowserModule}. Jak można zauważyć oba rodzaje modułów bardzo dobrze ze sobą współgrają.
	
	\begin{code}[
		language=javascript,
		caption={Stosowanie modułów Angular (źródło: \cite{javascript-module})},
		label={lis:angular-module},
		escapechar=|
	]
import { BrowserModule } from '@angular/platform-browser';
import { NgModule } from '@angular/core';
import { AppComponent } from './app.component';

@NgModule({
  declarations: [
    AppComponent
  ],
  imports: [
    BrowserModule
  ],
  providers: [],
  bootstrap: [AppComponent]
})
export class AppModule { }
	\end{code}\\
	
	Angular jako framework zawiera w sobie wiele gotowych do użycia modułów, których funkcjonalności przedstawiono w tabelce \ref{tab:angular-frequent-modules}.
	
	\begin{center}
    	\begin{table}[ht]
    	\begin{tabular}{ | l | l | p{5.5cm} |}
    \hline
    NgModule & Lokalizacja & Zastosowania \\ \hline
    BrowserModule & @angular/platform-browser & Kiedy aplikacja jest uruchamiana w przeglądarce \\ \hline
	CommonModule & @angular/common & Użycie dyrektyw takich jak: NgIf, NgFor, NgSwitch, NgClass, itp. \\ \hline
	FormsModule & @angular/forms & Podczas tworzenia formularzy opartych na szablonach \\ \hline
	ReactiveFormsModule & @angular/forms & Podczas tworzenia rektywnych formularzy \\ \hline
	RouterModule & @angular/router & Routing \\ \hline
	HttpClientModule & @angular/common/http & Wysyłanie zapytań HTTP typu AJAX \\ \hline
    \end{tabular}
    	\caption{Popularne moduły dołączone do Angular i ich zastosowania (źródło: \cite{angular-frequent-modules})}
    	\label{tab:angular-frequent-modules}
    	\end{table}
	\end{center}
	
	\subsection{Dołączanie danych}
	Domyślnie każde dołączenie danych jest jednokierunkowe, dzięki czemu maleje złożoność detekcji zmian. Zdecydowano się na podział na 3 sposoby:
	
	\subsubsection*{Jednokierunkowe ze źródła do celu}
	Stosowane przy dołączaniu do pól komponentu, dodawaniu dodatkowych atrybutów, takich jak \textit{aria-label}, dodawaniu klas, czy stylów lokalnych (ang. \textit{inline}). Składnię i przykładowe użycie przedstawiono w kodzie źródłowym \ref{lis:angular-one-way-binding-to-component}.
	
	\begin{code}[
		language=javascript,
		caption={Składnia i użycie dołączenia jednokierunkowego ze źródła do celu (źródło: \cite{angular-template-syntax})},
		label={lis:angular-one-way-binding-to-component},
		escapechar=|,
		keywords={},
	]
// Syntax
{{expression}}
[target]="expression"
bind-target="expression"

// Use
<img [src]="heroImageUrl">
<button [attr.aria-label]="help">help</button>
<div [class.special]="isSpecial">Special</div>
<button [style.color]="isSpecial ? 'red' : 'green'">
	\end{code}
	
	\subsubsection*{Jednokierunkowe od celu do źródła}
	Stosowane podczas reagowania na wydarzenia.	Jako wartość podaje się wyrażenie, które zostanie wywołane dopiero po jego zajściu. Dlatego istnieje bezpośrednia możliwość wywołania funkcji z odpowiednimi parametrami, zamiast przekazywania jej referencji. Składnię i przykładowe użycie przedstawiono w kodzie źródłowym \ref{lis:angular-one-way-binding-from-component}.
	
	\begin{code}[
		language=javascript,
		caption={Składnia i użycie dołączenia jednokierunkowego od celu do źródła (źródło: \cite{angular-template-syntax})},
		label={lis:angular-one-way-binding-from-component},
		escapechar=|,
		keywords={},
	]
// Syntax
(target)="statement"
on-target="statement"

// Use
<button (click)="onSave()">Save</button>
<div (myClick)="clicked=\$event" clickable>click me</div>
	\end{code}
	
	\subsubsection*{Dwukierunkowe}
	Stosowane, gdy istnieje potrzeba wyświetlenia i aktualizowania danych, gdy użytkownik przeprowadzi zmiany. Najczęstszą sytuacją jest dołączenie wartości do elementu HTML \textit{<input/>}, której aktualna wartość powinna być pokazana w polu tekstowym, a jej zmiana powinna być odzwierciedlona w stanie aplikacji. Składnię i przykładowe użycie przedstawiono w kodzie źródłowym \ref{lis:angular-two-way-binding}.
	
	\begin{code}[
		language=javascript,
		caption={Składnia i użycie dołączenia dwukierunkowego (źródło: \cite{angular-template-syntax})},
		label={lis:angular-two-way-binding},
		escapechar=|,
		keywords={},
	]
// Syntax
[(target)]="expression"
bindon-target="expression"

// Use
<input [(ngModel)]="name">
	\end{code}
	
	\subsection{Komponenty}
	Tak jak w AngularJS pozostano przy koncepcji komponentów i dyrektyw. Jednak w tym przypadku to komponenty zostały głównymi strukturami zapewniającymi reużywalność powtarzających się części kodu. Tworzenie komponentów odbywa się za pomocą dekorowania klas dekoratorem \textit{@Component} lub dla dyrektywy \textit{@Directive}. Jako medatadane można przekazać, na przykład: listę animacji komponentu, nazwę selektora, atrybuty wychodzące, sposób detekcji zmian i wiele innych. W kodzie źródłowym \ref{lis:angular-component} przedstawiony został komponent oferujący taką funkcjonalność, jak dyrektywa pokazana w podrozdziale \ref{page:angularjs-directives-and-components} \textit{Dyrektywy i komponenty}.
	
	\begin{code}[
		language=javascript,
		caption={Użycie komponentu w Angularze (źródło: opracowanie własne)},
		label={lis:angular-component},
		escapechar=|
	]
// customer.ts
@Component({
  selector: 'my-customer',
  template: `Name: {{customer.name}}, Address: {{customer.address}}`
})
export class CustomerComponent {
  customer = {
    name: 'Adam',
    address: 'Belveder st.'
  }
}
  
// app.html
<my-customer></my-customer> <!-- Name: Adam, Address: Belveder st. -->
	\end{code}
	
	\subsubsection{Dane wejściowe i wyjściowe}
	Jednak komponent powinien być elastyczny i pozwalać na ponowne użycie za pomocą danych wejściowych i wyjściowych. W tym celu ponownie zastosowano metodę dekorowania, w tym wypadku pól klas komponentów lub dyrektyw, dekoratorami \textit{@Input} i \textit{@Output}. W kodzie źródłowym \ref{lis:angular-component-input-output-create} pokazany został przykład tego samego komponentu przyjmującego obiekt klienta jako dane wejściowe (linijka \ref{line:angular-input}) oraz emitującego dane wyjściowe (linijka \ref{line:angular-output-use}). W tym wypadku jest to napis emitowany co 1 sekundę.\par
	W kodzie źródłowym \ref{lis:angular-component-input-output-use} przedstawione zostało użycie  komponentu \textit{CustomerComponent} jako elementu HTML (linijka \ref{line:angular-component-element}). Warto zwrócić uwagę na sposób przekazywania danych jako atrybuty. W atrybucie \textit{[customer]} nawiasy klamrowe informują kompilator, że dane mają zostać dołączone jednokierunkowo do komponentu do polu o tej samej nazwie. W atrybucie \textit{(helloEmitter)} nawiasy klamrowe informują kompilator, że funkcja \textit{greet} powinna być wywołana za każdym razem kiedy nastąpi emisja danych, a jej parametrem ma być ta emitowana wiadomość. Takie oddzielenie składni pozwala na łatwe dostrzeżenie danych wchodzących, jak i wychodzących.

	\begin{code}[
		language=javascript,
		caption={Utworzenie komponentu z danymi wejściowymi Angularze (źródło: opracowanie własne)},
		label={lis:angular-component-input-output-create},
		escapechar=|
	]
// customer.ts
export interface User {
  name: string;
  address: string;
}
@Component({
  selector: 'my-customer',
  template: `Name: {{customer.name}}, Address: {{customer.address}}`
})
export class CustomerComponent implements OnInit, OnDestroy {
  @Input() customer: User; |\label{line:angular-input}|
  @Output() helloEmitter = new EventEmitter<string>();
  private intervalId;

  ngOnInit(): void { |\label{line:angular-ngOnInit}|
    this.intervalId = setInterval(() => {
      this.helloEmitter.emit('Hello: ' + this.customer.name); |\label{line:angular-output-use}|
    }, 1000);
  }
  ngOnDestroy(): void { |\label{line:angular-ngOnDestroy}|
    clearInterval(this.intervalId);
  }
}
	\end{code}
	
	\begin{code}[
		language=javascript,
		caption={Użycie komponentu z danymi wejściowymi Angularze (źródło: opracowanie własne)},
		label={lis:angular-component-input-output-use},
		escapechar=|
	]
// app.ts
@Component({
  selector: 'app-root',
  template: `
<my-customer |\label{line:angular-component-element}|
  [customer]="myCustomer"
  (helloEmitter)="greet(\$event)"
></my-customer> 
  `
})
export class AppComponent {
  myCustomer: User = {
    name: 'Adam',
    address: 'Belveder st.'
  };
  greet(message: string) {
    console.log(message);
  }
}
	\end{code}
	
	\subsubsection{Cykl życia}
	Każdy komponent ma własny cykl życia zarządzany przez framework \cite{angular-lifecycle-hooks}. Angular tworzy komponent, renderuje go, tworzy i renderuje jego dzieci, sprawdza zamianę danych wejściowych i niszczy go przed usunięciem z drzewa DOM. Istnieje możliwość implementacji  jednej lub wielu funkcji wywoływanych właśnie w tych kluczowych momentach. Tak samo zachowują się również dyrektywy.\par
	Przykład użycia zaprezentowano w kodzie źródłowym \ref{lis:angular-component-input-output-create} w linijkach \ref{line:angular-ngOnInit} i \ref{line:angular-ngOnDestroy}. W metodzie \textit{ngOnInit()}, która wywoływana jest tuż po utworzeniu komponentu, uruchomiony zostaje licznik emitujący co sekundę napis. W metodzie \textit{ngOnDestroy()}, która wywoływana jest tuż przed usunięciem komponentu licznik ten zostaje zatrzymany w celu uniknięcia wycieków pamięci.
	
	\subsection{Dyrektywy}
	W Angularze istnieją 2 rodzaje dyrektyw: atrybutowe i strukturalne. Używa się ich za pomocą atrybutów elementów HTML. Atrybutowe \cite{angular-attribute-directives} zmieniają wygląd lub zachowanie elementu drzewa DOM.  Natomiast strukturalne \cite{angular-structural-directives} manipulują drzewem DOM.\par
	Angular posiada moduł z kilkoma bardzo przydatnymi dyrektywami strukturalnymi. W kodzie źródłowym \ref{lis:angular-structural-directives} w linijce \ref{line:angular-ngif} użyta została dyrektywa \textit{ngIf}, która usuwa z drzewa DOM element, na którym została użyta, jeśli wyrażenie użyte jako jej wartość jest fałszywe. W linijce \ref{line:angular-ngfor} użyta zostałą dyrektywa \textit{ngFor}, która powiela element, na którym została użyta, tyle razy, ile istnieje elementów w tablicy \textit{customers}. Składnia wyrażenia przypomina tą z pętli \textit{for/of} w ECMAScript 6. Warto zauważyć, że użycie dyrektyw strukturalnych jako atrybutów poprzedzone jest gwiazdką (\textit{*}).
	
	\begin{code}[
		language=javascript,
		caption={Użycie wbudowanych dyrektyw strukturalnych w Angularze (źródło: opracowanie własne)},
		label={lis:angular-structural-directives},
		escapechar=|
	]
<div *ngIf="age >= 18"> |\label{line:angular-ngif}|
  Adult
</div>

<div *ngFor="let c of customers"> |\label{line:angular-ngfor}|
  {{c.name}}
</div>
	\end{code}
	
	\subsection{Filtry}
	Zdecydowano się również na zachowanie Filtrów, choć teraz pod nazwą \textit{Pipes}. Ich rola pozostała niezmienna, lecz zmienił się tylko sposób ich tworzenia, tak jak pokazano to w kodzie źródłowym \ref{lis:angular-filter}.
	
	\begin{code}[
		language=javascript,
		caption={Utworzenie i użycie filtru w Angularze (źródło: opracowanie własne)},
		label={lis:angular-filter}
	]
// incrementer.pipe.ts
@Pipe({
  name: 'incrementer'
})
export class IncrementerPipe implements PipeTransform {
  transform(value: any, ...args: any[]): any {
    return value + 1;
  }
}

// app.html
<div>
	<p>{{5 | incrementer}}</p> <!-- 6 -->
</div>
	\end{code} 

	\subsection{Serwisy i wstrzykiwanie zależności}
	Tak jak AngularJS zdecydowano się na zastosowanie wstrzykiwania zależności. Dostarczenie zależności odbywa się poprzez przekazanie ich jako parametry konstruktora. Dlatego właśnie podczas deklaracji komponentów w modułach przekazuje się ich referencje, a nie utworzone już obiekty. Angular sam znajdzie i zdecyduje które obiekty użyć, a w przypadku ich braku wyrzuci odpowiedni wyjątek.\par
	Wstrzyknięcie odpowiednich obiektów można zawdzięczyć TypeScriptowi. Tak jak pokazano to w kodzie źródłowym \ref{lis:angular-serwis} przy parametrach konstruktora należy podać typ oczekiwanej wartości. Następnie za pomocą refleksji Angular jest w stanie wykryć jakie obiekty i w jakiej kolejności są potrzebne. Oczywiście istnieje możliwość zależności cyklicznej, czyli takiej, w której komponenty, czy serwisy są zależne od siebie nawzajem. W takim wypadku wstrzyknięcie zależności powinno odbyć się poprzez wstrzyknięcie serwisu \textit{Injector} a następnie wywołaniu na nim metody \textit{get()} z typem zależności cyklicznej.\par
	Podobnie jak w AngularJS zastosowano koncepcję serwisów. Każda klasa serwisu musi zostać opisana dekoratorem \textit{@Injectable()} a następnie jej referencja przekazana do modułu w tablicy \textit{providers}. Po tych krokach serwis dostępny jest do wykorzystania pozostałym strukturom tego modułu. W kodzie źródłowym \ref{lis:angular-serwis} przedstawiono zachowanie takie jak w kodzie źródłowym \ref{lis:angularjs-serwis}. 
	
	\begin{code}[
		language=javascript,
		caption={Utworzenie i użycie serwisu w Angularze (źródło: opracowanie własne)},
		label={lis:angular-serwis},
		escapechar=|
	]
@Injectable()
export class DateService {
  getCurrentDate(): Date {
    return new Date();
  }
}
@Component({selector: 'my-component', template: ''})
export class MyComponent implements OnInit {
  constructor(public dateService: DateService) {
  }
  ngOnInit(): void {
    console.log(this.dateService.getCurrentDate());
  }
}
@NgModule({
  providers: [DateService],
  declarations: [MyComponent]
})
export class MyModule {
}
	\end{code} 
	
	\subsection{Router}
	Angular posiada również moduł odpowiedzialny za płynne nawigowanie między widokami. W kodzie źródłowym \ref{lis:angular-router} przedstawiona została konfiguracja routera identyczna w stosunku do tej z kodu źródłowego \ref{lis:angularjs-router}. Obiekt konfiguracyjny należy przekazać do statycznej metody \textit{forRoot} w trakcie importowania modułu do aplikacji.
	
	\begin{code}[
		language=javascript,
		caption={Konfiguracja routera w Angularze (źródło: opracowanie własne)},
		label={lis:angular-router},
		escapechar=|
	]
const appRoutes: Routes = [
  {path: '/shop', component: ShopComponent},
  {path: '/shop/item/:itemId', component: ShopItemComponent},
];

@NgModule({
  imports: [
    RouterModule.forRoot(
      appRoutes,
    )
  ],
})
export class AppModule {
}
	\end{code} 
	


























